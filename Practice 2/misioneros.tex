
% Default to the notebook output style

    


% Inherit from the specified cell style.




    
\documentclass[11pt]{article}

    
    
    \usepackage[T1]{fontenc}
    % Nicer default font (+ math font) than Computer Modern for most use cases
    \usepackage{mathpazo}

    % Basic figure setup, for now with no caption control since it's done
    % automatically by Pandoc (which extracts ![](path) syntax from Markdown).
    \usepackage{graphicx}
    % We will generate all images so they have a width \maxwidth. This means
    % that they will get their normal width if they fit onto the page, but
    % are scaled down if they would overflow the margins.
    \makeatletter
    \def\maxwidth{\ifdim\Gin@nat@width>\linewidth\linewidth
    \else\Gin@nat@width\fi}
    \makeatother
    \let\Oldincludegraphics\includegraphics
    % Set max figure width to be 80% of text width, for now hardcoded.
    \renewcommand{\includegraphics}[1]{\Oldincludegraphics[width=.8\maxwidth]{#1}}
    % Ensure that by default, figures have no caption (until we provide a
    % proper Figure object with a Caption API and a way to capture that
    % in the conversion process - todo).
    \usepackage{caption}
    \DeclareCaptionLabelFormat{nolabel}{}
    \captionsetup{labelformat=nolabel}

    \usepackage{adjustbox} % Used to constrain images to a maximum size 
    \usepackage{xcolor} % Allow colors to be defined
    \usepackage{enumerate} % Needed for markdown enumerations to work
    \usepackage{geometry} % Used to adjust the document margins
    \usepackage{amsmath} % Equations
    \usepackage{amssymb} % Equations
    \usepackage{textcomp} % defines textquotesingle
    % Hack from http://tex.stackexchange.com/a/47451/13684:
    \AtBeginDocument{%
        \def\PYZsq{\textquotesingle}% Upright quotes in Pygmentized code
    }
    \usepackage{upquote} % Upright quotes for verbatim code
    \usepackage{eurosym} % defines \euro
    \usepackage[mathletters]{ucs} % Extended unicode (utf-8) support
    \usepackage[utf8x]{inputenc} % Allow utf-8 characters in the tex document
    \usepackage{fancyvrb} % verbatim replacement that allows latex
    \usepackage{grffile} % extends the file name processing of package graphics 
                         % to support a larger range 
    % The hyperref package gives us a pdf with properly built
    % internal navigation ('pdf bookmarks' for the table of contents,
    % internal cross-reference links, web links for URLs, etc.)
    \usepackage{hyperref}
    \usepackage{longtable} % longtable support required by pandoc >1.10
    \usepackage{booktabs}  % table support for pandoc > 1.12.2
    \usepackage[inline]{enumitem} % IRkernel/repr support (it uses the enumerate* environment)
    \usepackage[normalem]{ulem} % ulem is needed to support strikethroughs (\sout)
                                % normalem makes italics be italics, not underlines
    

    
    
    % Colors for the hyperref package
    \definecolor{urlcolor}{rgb}{0,.145,.698}
    \definecolor{linkcolor}{rgb}{.71,0.21,0.01}
    \definecolor{citecolor}{rgb}{.12,.54,.11}

    % ANSI colors
    \definecolor{ansi-black}{HTML}{3E424D}
    \definecolor{ansi-black-intense}{HTML}{282C36}
    \definecolor{ansi-red}{HTML}{E75C58}
    \definecolor{ansi-red-intense}{HTML}{B22B31}
    \definecolor{ansi-green}{HTML}{00A250}
    \definecolor{ansi-green-intense}{HTML}{007427}
    \definecolor{ansi-yellow}{HTML}{DDB62B}
    \definecolor{ansi-yellow-intense}{HTML}{B27D12}
    \definecolor{ansi-blue}{HTML}{208FFB}
    \definecolor{ansi-blue-intense}{HTML}{0065CA}
    \definecolor{ansi-magenta}{HTML}{D160C4}
    \definecolor{ansi-magenta-intense}{HTML}{A03196}
    \definecolor{ansi-cyan}{HTML}{60C6C8}
    \definecolor{ansi-cyan-intense}{HTML}{258F8F}
    \definecolor{ansi-white}{HTML}{C5C1B4}
    \definecolor{ansi-white-intense}{HTML}{A1A6B2}

    % commands and environments needed by pandoc snippets
    % extracted from the output of `pandoc -s`
    \providecommand{\tightlist}{%
      \setlength{\itemsep}{0pt}\setlength{\parskip}{0pt}}
    \DefineVerbatimEnvironment{Highlighting}{Verbatim}{commandchars=\\\{\}}
    % Add ',fontsize=\small' for more characters per line
    \newenvironment{Shaded}{}{}
    \newcommand{\KeywordTok}[1]{\textcolor[rgb]{0.00,0.44,0.13}{\textbf{{#1}}}}
    \newcommand{\DataTypeTok}[1]{\textcolor[rgb]{0.56,0.13,0.00}{{#1}}}
    \newcommand{\DecValTok}[1]{\textcolor[rgb]{0.25,0.63,0.44}{{#1}}}
    \newcommand{\BaseNTok}[1]{\textcolor[rgb]{0.25,0.63,0.44}{{#1}}}
    \newcommand{\FloatTok}[1]{\textcolor[rgb]{0.25,0.63,0.44}{{#1}}}
    \newcommand{\CharTok}[1]{\textcolor[rgb]{0.25,0.44,0.63}{{#1}}}
    \newcommand{\StringTok}[1]{\textcolor[rgb]{0.25,0.44,0.63}{{#1}}}
    \newcommand{\CommentTok}[1]{\textcolor[rgb]{0.38,0.63,0.69}{\textit{{#1}}}}
    \newcommand{\OtherTok}[1]{\textcolor[rgb]{0.00,0.44,0.13}{{#1}}}
    \newcommand{\AlertTok}[1]{\textcolor[rgb]{1.00,0.00,0.00}{\textbf{{#1}}}}
    \newcommand{\FunctionTok}[1]{\textcolor[rgb]{0.02,0.16,0.49}{{#1}}}
    \newcommand{\RegionMarkerTok}[1]{{#1}}
    \newcommand{\ErrorTok}[1]{\textcolor[rgb]{1.00,0.00,0.00}{\textbf{{#1}}}}
    \newcommand{\NormalTok}[1]{{#1}}
    
    % Additional commands for more recent versions of Pandoc
    \newcommand{\ConstantTok}[1]{\textcolor[rgb]{0.53,0.00,0.00}{{#1}}}
    \newcommand{\SpecialCharTok}[1]{\textcolor[rgb]{0.25,0.44,0.63}{{#1}}}
    \newcommand{\VerbatimStringTok}[1]{\textcolor[rgb]{0.25,0.44,0.63}{{#1}}}
    \newcommand{\SpecialStringTok}[1]{\textcolor[rgb]{0.73,0.40,0.53}{{#1}}}
    \newcommand{\ImportTok}[1]{{#1}}
    \newcommand{\DocumentationTok}[1]{\textcolor[rgb]{0.73,0.13,0.13}{\textit{{#1}}}}
    \newcommand{\AnnotationTok}[1]{\textcolor[rgb]{0.38,0.63,0.69}{\textbf{\textit{{#1}}}}}
    \newcommand{\CommentVarTok}[1]{\textcolor[rgb]{0.38,0.63,0.69}{\textbf{\textit{{#1}}}}}
    \newcommand{\VariableTok}[1]{\textcolor[rgb]{0.10,0.09,0.49}{{#1}}}
    \newcommand{\ControlFlowTok}[1]{\textcolor[rgb]{0.00,0.44,0.13}{\textbf{{#1}}}}
    \newcommand{\OperatorTok}[1]{\textcolor[rgb]{0.40,0.40,0.40}{{#1}}}
    \newcommand{\BuiltInTok}[1]{{#1}}
    \newcommand{\ExtensionTok}[1]{{#1}}
    \newcommand{\PreprocessorTok}[1]{\textcolor[rgb]{0.74,0.48,0.00}{{#1}}}
    \newcommand{\AttributeTok}[1]{\textcolor[rgb]{0.49,0.56,0.16}{{#1}}}
    \newcommand{\InformationTok}[1]{\textcolor[rgb]{0.38,0.63,0.69}{\textbf{\textit{{#1}}}}}
    \newcommand{\WarningTok}[1]{\textcolor[rgb]{0.38,0.63,0.69}{\textbf{\textit{{#1}}}}}
    
    
    % Define a nice break command that doesn't care if a line doesn't already
    % exist.
    \def\br{\hspace*{\fill} \\* }
    % Math Jax compatability definitions
    \def\gt{>}
    \def\lt{<}
    % Document parameters
    \title{misioneros}
    
    
    

    % Pygments definitions
    
\makeatletter
\def\PY@reset{\let\PY@it=\relax \let\PY@bf=\relax%
    \let\PY@ul=\relax \let\PY@tc=\relax%
    \let\PY@bc=\relax \let\PY@ff=\relax}
\def\PY@tok#1{\csname PY@tok@#1\endcsname}
\def\PY@toks#1+{\ifx\relax#1\empty\else%
    \PY@tok{#1}\expandafter\PY@toks\fi}
\def\PY@do#1{\PY@bc{\PY@tc{\PY@ul{%
    \PY@it{\PY@bf{\PY@ff{#1}}}}}}}
\def\PY#1#2{\PY@reset\PY@toks#1+\relax+\PY@do{#2}}

\expandafter\def\csname PY@tok@w\endcsname{\def\PY@tc##1{\textcolor[rgb]{0.73,0.73,0.73}{##1}}}
\expandafter\def\csname PY@tok@c\endcsname{\let\PY@it=\textit\def\PY@tc##1{\textcolor[rgb]{0.25,0.50,0.50}{##1}}}
\expandafter\def\csname PY@tok@cp\endcsname{\def\PY@tc##1{\textcolor[rgb]{0.74,0.48,0.00}{##1}}}
\expandafter\def\csname PY@tok@k\endcsname{\let\PY@bf=\textbf\def\PY@tc##1{\textcolor[rgb]{0.00,0.50,0.00}{##1}}}
\expandafter\def\csname PY@tok@kp\endcsname{\def\PY@tc##1{\textcolor[rgb]{0.00,0.50,0.00}{##1}}}
\expandafter\def\csname PY@tok@kt\endcsname{\def\PY@tc##1{\textcolor[rgb]{0.69,0.00,0.25}{##1}}}
\expandafter\def\csname PY@tok@o\endcsname{\def\PY@tc##1{\textcolor[rgb]{0.40,0.40,0.40}{##1}}}
\expandafter\def\csname PY@tok@ow\endcsname{\let\PY@bf=\textbf\def\PY@tc##1{\textcolor[rgb]{0.67,0.13,1.00}{##1}}}
\expandafter\def\csname PY@tok@nb\endcsname{\def\PY@tc##1{\textcolor[rgb]{0.00,0.50,0.00}{##1}}}
\expandafter\def\csname PY@tok@nf\endcsname{\def\PY@tc##1{\textcolor[rgb]{0.00,0.00,1.00}{##1}}}
\expandafter\def\csname PY@tok@nc\endcsname{\let\PY@bf=\textbf\def\PY@tc##1{\textcolor[rgb]{0.00,0.00,1.00}{##1}}}
\expandafter\def\csname PY@tok@nn\endcsname{\let\PY@bf=\textbf\def\PY@tc##1{\textcolor[rgb]{0.00,0.00,1.00}{##1}}}
\expandafter\def\csname PY@tok@ne\endcsname{\let\PY@bf=\textbf\def\PY@tc##1{\textcolor[rgb]{0.82,0.25,0.23}{##1}}}
\expandafter\def\csname PY@tok@nv\endcsname{\def\PY@tc##1{\textcolor[rgb]{0.10,0.09,0.49}{##1}}}
\expandafter\def\csname PY@tok@no\endcsname{\def\PY@tc##1{\textcolor[rgb]{0.53,0.00,0.00}{##1}}}
\expandafter\def\csname PY@tok@nl\endcsname{\def\PY@tc##1{\textcolor[rgb]{0.63,0.63,0.00}{##1}}}
\expandafter\def\csname PY@tok@ni\endcsname{\let\PY@bf=\textbf\def\PY@tc##1{\textcolor[rgb]{0.60,0.60,0.60}{##1}}}
\expandafter\def\csname PY@tok@na\endcsname{\def\PY@tc##1{\textcolor[rgb]{0.49,0.56,0.16}{##1}}}
\expandafter\def\csname PY@tok@nt\endcsname{\let\PY@bf=\textbf\def\PY@tc##1{\textcolor[rgb]{0.00,0.50,0.00}{##1}}}
\expandafter\def\csname PY@tok@nd\endcsname{\def\PY@tc##1{\textcolor[rgb]{0.67,0.13,1.00}{##1}}}
\expandafter\def\csname PY@tok@s\endcsname{\def\PY@tc##1{\textcolor[rgb]{0.73,0.13,0.13}{##1}}}
\expandafter\def\csname PY@tok@sd\endcsname{\let\PY@it=\textit\def\PY@tc##1{\textcolor[rgb]{0.73,0.13,0.13}{##1}}}
\expandafter\def\csname PY@tok@si\endcsname{\let\PY@bf=\textbf\def\PY@tc##1{\textcolor[rgb]{0.73,0.40,0.53}{##1}}}
\expandafter\def\csname PY@tok@se\endcsname{\let\PY@bf=\textbf\def\PY@tc##1{\textcolor[rgb]{0.73,0.40,0.13}{##1}}}
\expandafter\def\csname PY@tok@sr\endcsname{\def\PY@tc##1{\textcolor[rgb]{0.73,0.40,0.53}{##1}}}
\expandafter\def\csname PY@tok@ss\endcsname{\def\PY@tc##1{\textcolor[rgb]{0.10,0.09,0.49}{##1}}}
\expandafter\def\csname PY@tok@sx\endcsname{\def\PY@tc##1{\textcolor[rgb]{0.00,0.50,0.00}{##1}}}
\expandafter\def\csname PY@tok@m\endcsname{\def\PY@tc##1{\textcolor[rgb]{0.40,0.40,0.40}{##1}}}
\expandafter\def\csname PY@tok@gh\endcsname{\let\PY@bf=\textbf\def\PY@tc##1{\textcolor[rgb]{0.00,0.00,0.50}{##1}}}
\expandafter\def\csname PY@tok@gu\endcsname{\let\PY@bf=\textbf\def\PY@tc##1{\textcolor[rgb]{0.50,0.00,0.50}{##1}}}
\expandafter\def\csname PY@tok@gd\endcsname{\def\PY@tc##1{\textcolor[rgb]{0.63,0.00,0.00}{##1}}}
\expandafter\def\csname PY@tok@gi\endcsname{\def\PY@tc##1{\textcolor[rgb]{0.00,0.63,0.00}{##1}}}
\expandafter\def\csname PY@tok@gr\endcsname{\def\PY@tc##1{\textcolor[rgb]{1.00,0.00,0.00}{##1}}}
\expandafter\def\csname PY@tok@ge\endcsname{\let\PY@it=\textit}
\expandafter\def\csname PY@tok@gs\endcsname{\let\PY@bf=\textbf}
\expandafter\def\csname PY@tok@gp\endcsname{\let\PY@bf=\textbf\def\PY@tc##1{\textcolor[rgb]{0.00,0.00,0.50}{##1}}}
\expandafter\def\csname PY@tok@go\endcsname{\def\PY@tc##1{\textcolor[rgb]{0.53,0.53,0.53}{##1}}}
\expandafter\def\csname PY@tok@gt\endcsname{\def\PY@tc##1{\textcolor[rgb]{0.00,0.27,0.87}{##1}}}
\expandafter\def\csname PY@tok@err\endcsname{\def\PY@bc##1{\setlength{\fboxsep}{0pt}\fcolorbox[rgb]{1.00,0.00,0.00}{1,1,1}{\strut ##1}}}
\expandafter\def\csname PY@tok@kc\endcsname{\let\PY@bf=\textbf\def\PY@tc##1{\textcolor[rgb]{0.00,0.50,0.00}{##1}}}
\expandafter\def\csname PY@tok@kd\endcsname{\let\PY@bf=\textbf\def\PY@tc##1{\textcolor[rgb]{0.00,0.50,0.00}{##1}}}
\expandafter\def\csname PY@tok@kn\endcsname{\let\PY@bf=\textbf\def\PY@tc##1{\textcolor[rgb]{0.00,0.50,0.00}{##1}}}
\expandafter\def\csname PY@tok@kr\endcsname{\let\PY@bf=\textbf\def\PY@tc##1{\textcolor[rgb]{0.00,0.50,0.00}{##1}}}
\expandafter\def\csname PY@tok@bp\endcsname{\def\PY@tc##1{\textcolor[rgb]{0.00,0.50,0.00}{##1}}}
\expandafter\def\csname PY@tok@fm\endcsname{\def\PY@tc##1{\textcolor[rgb]{0.00,0.00,1.00}{##1}}}
\expandafter\def\csname PY@tok@vc\endcsname{\def\PY@tc##1{\textcolor[rgb]{0.10,0.09,0.49}{##1}}}
\expandafter\def\csname PY@tok@vg\endcsname{\def\PY@tc##1{\textcolor[rgb]{0.10,0.09,0.49}{##1}}}
\expandafter\def\csname PY@tok@vi\endcsname{\def\PY@tc##1{\textcolor[rgb]{0.10,0.09,0.49}{##1}}}
\expandafter\def\csname PY@tok@vm\endcsname{\def\PY@tc##1{\textcolor[rgb]{0.10,0.09,0.49}{##1}}}
\expandafter\def\csname PY@tok@sa\endcsname{\def\PY@tc##1{\textcolor[rgb]{0.73,0.13,0.13}{##1}}}
\expandafter\def\csname PY@tok@sb\endcsname{\def\PY@tc##1{\textcolor[rgb]{0.73,0.13,0.13}{##1}}}
\expandafter\def\csname PY@tok@sc\endcsname{\def\PY@tc##1{\textcolor[rgb]{0.73,0.13,0.13}{##1}}}
\expandafter\def\csname PY@tok@dl\endcsname{\def\PY@tc##1{\textcolor[rgb]{0.73,0.13,0.13}{##1}}}
\expandafter\def\csname PY@tok@s2\endcsname{\def\PY@tc##1{\textcolor[rgb]{0.73,0.13,0.13}{##1}}}
\expandafter\def\csname PY@tok@sh\endcsname{\def\PY@tc##1{\textcolor[rgb]{0.73,0.13,0.13}{##1}}}
\expandafter\def\csname PY@tok@s1\endcsname{\def\PY@tc##1{\textcolor[rgb]{0.73,0.13,0.13}{##1}}}
\expandafter\def\csname PY@tok@mb\endcsname{\def\PY@tc##1{\textcolor[rgb]{0.40,0.40,0.40}{##1}}}
\expandafter\def\csname PY@tok@mf\endcsname{\def\PY@tc##1{\textcolor[rgb]{0.40,0.40,0.40}{##1}}}
\expandafter\def\csname PY@tok@mh\endcsname{\def\PY@tc##1{\textcolor[rgb]{0.40,0.40,0.40}{##1}}}
\expandafter\def\csname PY@tok@mi\endcsname{\def\PY@tc##1{\textcolor[rgb]{0.40,0.40,0.40}{##1}}}
\expandafter\def\csname PY@tok@il\endcsname{\def\PY@tc##1{\textcolor[rgb]{0.40,0.40,0.40}{##1}}}
\expandafter\def\csname PY@tok@mo\endcsname{\def\PY@tc##1{\textcolor[rgb]{0.40,0.40,0.40}{##1}}}
\expandafter\def\csname PY@tok@ch\endcsname{\let\PY@it=\textit\def\PY@tc##1{\textcolor[rgb]{0.25,0.50,0.50}{##1}}}
\expandafter\def\csname PY@tok@cm\endcsname{\let\PY@it=\textit\def\PY@tc##1{\textcolor[rgb]{0.25,0.50,0.50}{##1}}}
\expandafter\def\csname PY@tok@cpf\endcsname{\let\PY@it=\textit\def\PY@tc##1{\textcolor[rgb]{0.25,0.50,0.50}{##1}}}
\expandafter\def\csname PY@tok@c1\endcsname{\let\PY@it=\textit\def\PY@tc##1{\textcolor[rgb]{0.25,0.50,0.50}{##1}}}
\expandafter\def\csname PY@tok@cs\endcsname{\let\PY@it=\textit\def\PY@tc##1{\textcolor[rgb]{0.25,0.50,0.50}{##1}}}

\def\PYZbs{\char`\\}
\def\PYZus{\char`\_}
\def\PYZob{\char`\{}
\def\PYZcb{\char`\}}
\def\PYZca{\char`\^}
\def\PYZam{\char`\&}
\def\PYZlt{\char`\<}
\def\PYZgt{\char`\>}
\def\PYZsh{\char`\#}
\def\PYZpc{\char`\%}
\def\PYZdl{\char`\$}
\def\PYZhy{\char`\-}
\def\PYZsq{\char`\'}
\def\PYZdq{\char`\"}
\def\PYZti{\char`\~}
% for compatibility with earlier versions
\def\PYZat{@}
\def\PYZlb{[}
\def\PYZrb{]}
\makeatother


    % Exact colors from NB
    \definecolor{incolor}{rgb}{0.0, 0.0, 0.5}
    \definecolor{outcolor}{rgb}{0.545, 0.0, 0.0}



    
    % Prevent overflowing lines due to hard-to-break entities
    \sloppy 
    % Setup hyperref package
    \hypersetup{
      breaklinks=true,  % so long urls are correctly broken across lines
      colorlinks=true,
      urlcolor=urlcolor,
      linkcolor=linkcolor,
      citecolor=citecolor,
      }
    % Slightly bigger margins than the latex defaults
    
    \geometry{verbose,tmargin=1in,bmargin=1in,lmargin=1in,rmargin=1in}
    
    

    \begin{document}
    
    
    \maketitle
    
    

    
    \subsection{EL PROBLEMA DE LOS MISIONEROS -\/-\/- PRÁCTICA 2 IA GRUPO
5}\label{el-problema-de-los-misioneros-----pruxe1ctica-2-ia-grupo-5}

    \subsubsection{Definición del problema siguiendo el esquema
proporcionado por
AIMA}\label{definiciuxf3n-del-problema-siguiendo-el-esquema-proporcionado-por-aima}

    \begin{Verbatim}[commandchars=\\\{\}]
{\color{incolor}In [{\color{incolor}1}]:} \PY{c+c1}{\PYZsh{} Importamos la librería search de AIMA}
        \PY{k+kn}{from} \PY{n+nn}{search} \PY{k}{import} \PY{o}{*}
\end{Verbatim}


    \begin{Verbatim}[commandchars=\\\{\}]
{\color{incolor}In [{\color{incolor}2}]:} \PY{c+c1}{\PYZsh{} Definición del problema de los misioneros según el esquema de AIMA}
        \PY{k}{class} \PY{n+nc}{ProblemaMisioneros}\PY{p}{(}\PY{n}{Problem}\PY{p}{)}\PY{p}{:}
            \PY{l+s+sd}{\PYZsq{}\PYZsq{}\PYZsq{} Clase problema (formalizacion de nuestro problema) siguiendo la}
        \PY{l+s+sd}{        estructura que aima espera que tengan los problemas.\PYZsq{}\PYZsq{}\PYZsq{}}
            \PY{k}{def} \PY{n+nf}{\PYZus{}\PYZus{}init\PYZus{}\PYZus{}}\PY{p}{(}\PY{n+nb+bp}{self}\PY{p}{,} \PY{n}{initial}\PY{p}{,} \PY{n}{goal}\PY{o}{=}\PY{k+kc}{None}\PY{p}{)}\PY{p}{:}
                \PY{l+s+sd}{\PYZsq{}\PYZsq{}\PYZsq{}Inicializacion de nuestro problema.\PYZsq{}\PYZsq{}\PYZsq{}}
                \PY{n}{Problem}\PY{o}{.}\PY{n+nf+fm}{\PYZus{}\PYZus{}init\PYZus{}\PYZus{}}\PY{p}{(}\PY{n+nb+bp}{self}\PY{p}{,} \PY{n}{initial}\PY{p}{,} \PY{n}{goal}\PY{p}{)}
                \PY{c+c1}{\PYZsh{} cada accion tiene un texto para identificar al operador y despues una tupla con la}
                \PY{c+c1}{\PYZsh{} cantidad de misioneros y canibales que se mueven en la canoa}
                \PY{n+nb+bp}{self}\PY{o}{.}\PY{n}{\PYZus{}actions} \PY{o}{=} \PY{p}{[}\PY{p}{(}\PY{l+s+s1}{\PYZsq{}}\PY{l+s+s1}{1c}\PY{l+s+s1}{\PYZsq{}}\PY{p}{,} \PY{p}{(}\PY{l+m+mi}{0}\PY{p}{,}\PY{l+m+mi}{1}\PY{p}{)}\PY{p}{)}\PY{p}{,} \PY{p}{(}\PY{l+s+s1}{\PYZsq{}}\PY{l+s+s1}{1m}\PY{l+s+s1}{\PYZsq{}}\PY{p}{,} \PY{p}{(}\PY{l+m+mi}{1}\PY{p}{,} \PY{l+m+mi}{0}\PY{p}{)}\PY{p}{)}\PY{p}{,} \PY{p}{(}\PY{l+s+s1}{\PYZsq{}}\PY{l+s+s1}{2c}\PY{l+s+s1}{\PYZsq{}}\PY{p}{,} \PY{p}{(}\PY{l+m+mi}{0}\PY{p}{,} \PY{l+m+mi}{2}\PY{p}{)}\PY{p}{)}\PY{p}{,} \PY{p}{(}\PY{l+s+s1}{\PYZsq{}}\PY{l+s+s1}{2m}\PY{l+s+s1}{\PYZsq{}}\PY{p}{,} \PY{p}{(}\PY{l+m+mi}{2}\PY{p}{,} \PY{l+m+mi}{0}\PY{p}{)}\PY{p}{)}\PY{p}{,} \PY{p}{(}\PY{l+s+s1}{\PYZsq{}}\PY{l+s+s1}{1m1c}\PY{l+s+s1}{\PYZsq{}}\PY{p}{,} \PY{p}{(}\PY{l+m+mi}{1}\PY{p}{,} \PY{l+m+mi}{1}\PY{p}{)}\PY{p}{)}\PY{p}{]}
        
            \PY{k}{def} \PY{n+nf}{actions}\PY{p}{(}\PY{n+nb+bp}{self}\PY{p}{,} \PY{n}{s}\PY{p}{)}\PY{p}{:}
                \PY{l+s+sd}{\PYZsq{}\PYZsq{}\PYZsq{}Devuelve las acciones validas para un estado.\PYZsq{}\PYZsq{}\PYZsq{}}
                \PY{c+c1}{\PYZsh{} las acciones validas para un estado son aquellas que al aplicarse}
                \PY{c+c1}{\PYZsh{} nos dejan en otro estado valido}
                \PY{k}{return} \PY{p}{[}\PY{n}{a} \PY{k}{for} \PY{n}{a} \PY{o+ow}{in} \PY{n+nb+bp}{self}\PY{o}{.}\PY{n}{\PYZus{}actions} \PY{k}{if} \PY{n+nb+bp}{self}\PY{o}{.}\PY{n}{\PYZus{}is\PYZus{}valid}\PY{p}{(}\PY{n+nb+bp}{self}\PY{o}{.}\PY{n}{result}\PY{p}{(}\PY{n}{s}\PY{p}{,} \PY{n}{a}\PY{p}{)}\PY{p}{)}\PY{p}{]}
        
            \PY{k}{def} \PY{n+nf}{\PYZus{}is\PYZus{}valid}\PY{p}{(}\PY{n+nb+bp}{self}\PY{p}{,} \PY{n}{s}\PY{p}{)}\PY{p}{:}
                \PY{l+s+sd}{\PYZsq{}\PYZsq{}\PYZsq{}Determina si un estado es valido o no.\PYZsq{}\PYZsq{}\PYZsq{}}
                \PY{c+c1}{\PYZsh{} un estado es valido si no hay mas canibales que misioneros en ninguna}
                \PY{c+c1}{\PYZsh{} orilla, y si las cantidades estan entre 0 y 3}
                \PY{k}{return} \PY{p}{(}\PY{n}{s}\PY{p}{[}\PY{l+m+mi}{0}\PY{p}{]} \PY{o}{\PYZgt{}}\PY{o}{=} \PY{n}{s}\PY{p}{[}\PY{l+m+mi}{1}\PY{p}{]} \PY{o+ow}{or} \PY{n}{s}\PY{p}{[}\PY{l+m+mi}{0}\PY{p}{]} \PY{o}{==} \PY{l+m+mi}{0}\PY{p}{)} \PY{o+ow}{and} \PY{p}{(}\PY{p}{(}\PY{l+m+mi}{3} \PY{o}{\PYZhy{}} \PY{n}{s}\PY{p}{[}\PY{l+m+mi}{0}\PY{p}{]}\PY{p}{)} \PY{o}{\PYZgt{}}\PY{o}{=} \PY{p}{(}\PY{l+m+mi}{3} \PY{o}{\PYZhy{}} \PY{n}{s}\PY{p}{[}\PY{l+m+mi}{1}\PY{p}{]}\PY{p}{)} \PY{o+ow}{or} \PY{n}{s}\PY{p}{[}\PY{l+m+mi}{0}\PY{p}{]} \PY{o}{==} \PY{l+m+mi}{3}\PY{p}{)} \PY{o+ow}{and} \PY{p}{(}\PY{l+m+mi}{0} \PY{o}{\PYZlt{}}\PY{o}{=} \PY{n}{s}\PY{p}{[}\PY{l+m+mi}{0}\PY{p}{]} \PY{o}{\PYZlt{}}\PY{o}{=} \PY{l+m+mi}{3}\PY{p}{)} \PY{o+ow}{and} \PY{p}{(}\PY{l+m+mi}{0} \PY{o}{\PYZlt{}}\PY{o}{=} \PY{n}{s}\PY{p}{[}\PY{l+m+mi}{1}\PY{p}{]} \PY{o}{\PYZlt{}}\PY{o}{=} \PY{l+m+mi}{3}\PY{p}{)}
        
            \PY{k}{def} \PY{n+nf}{result}\PY{p}{(}\PY{n+nb+bp}{self}\PY{p}{,} \PY{n}{s}\PY{p}{,} \PY{n}{a}\PY{p}{)}\PY{p}{:}
                \PY{l+s+sd}{\PYZsq{}\PYZsq{}\PYZsq{}Devuelve el estado resultante de aplicar una accion a un estado}
        \PY{l+s+sd}{           determinado.\PYZsq{}\PYZsq{}\PYZsq{}}
                \PY{c+c1}{\PYZsh{} el estado resultante tiene la canoa en el lado opuesto, y con las}
                \PY{c+c1}{\PYZsh{} cantidades de misioneros y canibales actualizadas segun la cantidad}
                \PY{c+c1}{\PYZsh{} que viajaron en la canoa}
                \PY{k}{if} \PY{n}{s}\PY{p}{[}\PY{l+m+mi}{2}\PY{p}{]} \PY{o}{==} \PY{l+m+mi}{0}\PY{p}{:}
                    \PY{k}{return} \PY{p}{(}\PY{n}{s}\PY{p}{[}\PY{l+m+mi}{0}\PY{p}{]} \PY{o}{\PYZhy{}} \PY{n}{a}\PY{p}{[}\PY{l+m+mi}{1}\PY{p}{]}\PY{p}{[}\PY{l+m+mi}{0}\PY{p}{]}\PY{p}{,} \PY{n}{s}\PY{p}{[}\PY{l+m+mi}{1}\PY{p}{]} \PY{o}{\PYZhy{}} \PY{n}{a}\PY{p}{[}\PY{l+m+mi}{1}\PY{p}{]}\PY{p}{[}\PY{l+m+mi}{1}\PY{p}{]}\PY{p}{,} \PY{l+m+mi}{1}\PY{p}{)}
                \PY{k}{else}\PY{p}{:}
                    \PY{k}{return} \PY{p}{(}\PY{n}{s}\PY{p}{[}\PY{l+m+mi}{0}\PY{p}{]} \PY{o}{+} \PY{n}{a}\PY{p}{[}\PY{l+m+mi}{1}\PY{p}{]}\PY{p}{[}\PY{l+m+mi}{0}\PY{p}{]}\PY{p}{,} \PY{n}{s}\PY{p}{[}\PY{l+m+mi}{1}\PY{p}{]} \PY{o}{+} \PY{n}{a}\PY{p}{[}\PY{l+m+mi}{1}\PY{p}{]}\PY{p}{[}\PY{l+m+mi}{1}\PY{p}{]}\PY{p}{,} \PY{l+m+mi}{0}\PY{p}{)}
\end{Verbatim}


    \begin{Verbatim}[commandchars=\\\{\}]
{\color{incolor}In [{\color{incolor}3}]:} \PY{c+c1}{\PYZsh{} Creación del problema dado un estado inicial y un estado objetivo}
        \PY{n}{estado} \PY{o}{=} \PY{n}{ProblemaMisioneros}\PY{p}{(}\PY{p}{(}\PY{l+m+mi}{3}\PY{p}{,} \PY{l+m+mi}{3}\PY{p}{,} \PY{l+m+mi}{0}\PY{p}{)}\PY{p}{,} \PY{p}{(}\PY{l+m+mi}{0}\PY{p}{,} \PY{l+m+mi}{0}\PY{p}{,} \PY{l+m+mi}{1}\PY{p}{)}\PY{p}{)}
\end{Verbatim}


    \subsubsection{a) Resolved el problema por los distintos métodos de
búsqueda
vistos:}\label{a-resolved-el-problema-por-los-distintos-muxe9todos-de-buxfasqueda-vistos}

    \begin{Verbatim}[commandchars=\\\{\}]
{\color{incolor}In [{\color{incolor}4}]:} \PY{c+c1}{\PYZsh{} Solución por búsqueda en anchura sin control de estados repetidos}
        \PY{n}{breadth\PYZus{}first\PYZus{}tree\PYZus{}search}\PY{p}{(}\PY{n}{estado}\PY{p}{)}\PY{o}{.}\PY{n}{solution}\PY{p}{(}\PY{p}{)}
\end{Verbatim}


\begin{Verbatim}[commandchars=\\\{\}]
{\color{outcolor}Out[{\color{outcolor}4}]:} [('2c', (0, 2)),
         ('1c', (0, 1)),
         ('2c', (0, 2)),
         ('1c', (0, 1)),
         ('2m', (2, 0)),
         ('1m1c', (1, 1)),
         ('2m', (2, 0)),
         ('1c', (0, 1)),
         ('2c', (0, 2)),
         ('1c', (0, 1)),
         ('2c', (0, 2))]
\end{Verbatim}
            
    \begin{Verbatim}[commandchars=\\\{\}]
{\color{incolor}In [{\color{incolor}5}]:} \PY{c+c1}{\PYZsh{} Solución por búsqueda en anchura con control de estados repetidos}
        \PY{n}{breadth\PYZus{}first\PYZus{}graph\PYZus{}search}\PY{p}{(}\PY{n}{estado}\PY{p}{)}\PY{o}{.}\PY{n}{solution}\PY{p}{(}\PY{p}{)}
\end{Verbatim}


\begin{Verbatim}[commandchars=\\\{\}]
{\color{outcolor}Out[{\color{outcolor}5}]:} [('2c', (0, 2)),
         ('1c', (0, 1)),
         ('2c', (0, 2)),
         ('1c', (0, 1)),
         ('2m', (2, 0)),
         ('1m1c', (1, 1)),
         ('2m', (2, 0)),
         ('1c', (0, 1)),
         ('2c', (0, 2)),
         ('1c', (0, 1)),
         ('2c', (0, 2))]
\end{Verbatim}
            
    \begin{Verbatim}[commandchars=\\\{\}]
{\color{incolor}In [{\color{incolor}6}]:} \PY{c+c1}{\PYZsh{} Solución por búsqueda en profundidad sin control de estados repetidos}
        \PY{n}{depth\PYZus{}first\PYZus{}graph\PYZus{}search}\PY{p}{(}\PY{n}{estado}\PY{p}{)}\PY{o}{.}\PY{n}{solution}\PY{p}{(}\PY{p}{)}
\end{Verbatim}


\begin{Verbatim}[commandchars=\\\{\}]
{\color{outcolor}Out[{\color{outcolor}6}]:} [('1m1c', (1, 1)),
         ('1m', (1, 0)),
         ('2c', (0, 2)),
         ('1c', (0, 1)),
         ('2m', (2, 0)),
         ('1m1c', (1, 1)),
         ('2m', (2, 0)),
         ('1c', (0, 1)),
         ('2c', (0, 2)),
         ('1m', (1, 0)),
         ('1m1c', (1, 1))]
\end{Verbatim}
            
    \begin{Verbatim}[commandchars=\\\{\}]
{\color{incolor}In [{\color{incolor}7}]:} \PY{c+c1}{\PYZsh{} \PYZhy{}\PYZhy{}\PYZhy{}\PYZhy{}\PYZhy{}\PYZhy{}\PYZhy{}\PYZhy{}\PYZhy{}\PYZhy{}\PYZhy{}\PYZhy{}\PYZhy{} NO FINALIZA \PYZhy{}\PYZhy{}\PYZhy{}\PYZhy{}\PYZhy{}\PYZhy{}\PYZhy{}\PYZhy{}\PYZhy{}\PYZhy{}\PYZhy{}\PYZhy{}\PYZhy{}\PYZhy{}}
        \PY{c+c1}{\PYZsh{} Solución por búsqueda en profundidad sin control de estados repetidos}
        \PY{c+c1}{\PYZsh{} depth\PYZus{}first\PYZus{}tree\PYZus{}search(estado).solution()}
\end{Verbatim}


    \subsubsection{b) Analizad el coste de las soluciones
encontradas:}\label{b-analizad-el-coste-de-las-soluciones-encontradas}

    \paragraph{Búsqueda en anchura sin control de
repetidos:}\label{buxfasqueda-en-anchura-sin-control-de-repetidos}

Encuentra una solución que requiere de once operaciones. Además,
suponiendo que cuando hablamos del coste nos referimos al número de
operaciones ejecutadas (con cada operación por tanto de coste uno), el
algoritmo de búsqueda en anchura alcanza siempre una solución óptima. La
solución dada requiere el mínimo número de operaciones para llegar al
estado objetivo.

    \paragraph{Búsqueda en anchura con control de
repetidos:}\label{buxfasqueda-en-anchura-con-control-de-repetidos}

Encuentra una solución que requiere de once operaciones, de hecho,
encuentra la misma que el algoritmo que no controla nodos repetidos.
Esto es lógico porque la naturaleza del algoritmo es la misma, completo,
por lo que siempre que exista solución encuentra una solución y óptimo
(gracias a que suponemos el coste igual al número de operaciones), por
lo que encuentra una solución que requiere el mínimo número de
operaciones.

¿Qué diferencia hallamos entonces entre estos dos algoritmos hermanos?
El primero al no requerir de estructuras de datos adicionales para el
control de nodos repetidos ni requerir de operaciones adicionales de
comprobación es un algoritmo válido e incluso más rápido cuando existen
soluciones con un bajo número de operaciones. Cuando este no es el caso,
el algoritmo analiza árboles de exploración enteros repetidas veces
(árboles que crecen exponencialmente) por lo que el coste en tiempo es
mayor.

    \paragraph{Búsqueda en profundidad sin control de
repetidos:}\label{buxfasqueda-en-profundidad-sin-control-de-repetidos}

En este caso el algoritmo no es capaz de hallar una solución al problema
ni de terminar su ejecución. Esto es debido a que al no existir un
control de nodos explorados y, más especialmente, a la existencia de
operaciones inversas el algoritmo puede caer en bucles infinitos
explorando ciclos.

Por ejemplo, la siguiente secuencia no atraviesa estados prohibidos y
genera un bucle:

S0 -\textgreater{} Mover dos caníbales derecha -\textgreater{} Mover dos
caníbales izquierda -\textgreater{} S0

    \paragraph{Búsqueda en profundidad con control de
repetidos:}\label{buxfasqueda-en-profundidad-con-control-de-repetidos}

Al añadir el control de nodos repetidos el algoritmo nunca examinará un
nodo ya explorado lo que impide que se creen bucles infinitos debidos a
la exploración de ciclos en el grafo de estados y operaciones de nuestro
problema. Con este algoritmo se halla una solución de coste once
operaciones por lo que sabemos que es óptima.

    \subsubsection{c) Analizad el coste en memoria de los distintos
algoritmos y la causa de sus
diferencias:}\label{c-analizad-el-coste-en-memoria-de-los-distintos-algoritmos-y-la-causa-de-sus-diferencias}

    \paragraph{-\/-\/-\/-\/-\/-\/-\/-\/-\/-\/-\/-\/-\/-\/-\/-\/-\/-\/-\/-\/-\/-\/-\/-\/-\/-\/-\/-\/-\/-\/-\/-\/-\/-\/-\/-\/-\/-\/-\/-\/-\/-\/-\/-\/-\/-\/-\/-\/-\/-\/-\/-\/-\/-\/-\/-\/-\/-\/-\/-\/-\/-\/-\/-\/-\/-\/-\/-\/-\/-\/-\/-\/-\/-\/-\/-\/-\/-\/-\/-\/-\/-\/-\/-\/-\/-\/-\/-\/-\/-\/-\/-\/-\/-\/-\/-\/-\/-\/-\/-\/-\/-\/-\/-\/-\/-\/-\/-\/-\/-\/-\/-\/-\/-\/-\/-\/-\/-\/-\/-\/-\/-\/-\/-\/-\/-\/-\/-\/-\/-\/-\/-\/-\/-\/-\/-\/-\/-\/-\/-\/-\/-\/-\/-\/-\/-\/-\/-\/-\/-\/-}\label{section}

    \begin{Verbatim}[commandchars=\\\{\}]
{\color{incolor}In [{\color{incolor}8}]:} \PY{c+c1}{\PYZsh{} Ampliamos la clase con atributos para realizar las métricas de memoria:}
        \PY{k}{class} \PY{n+nc}{ProblemaConMetricas}\PY{p}{(}\PY{n}{Problem}\PY{p}{)}\PY{p}{:}
        
            \PY{l+s+sd}{\PYZdq{}\PYZdq{}\PYZdq{}Clase extendida para incorporar atributos para la medida del coste de ejecución\PYZdq{}\PYZdq{}\PYZdq{}} 
                 
            \PY{k}{def} \PY{n+nf}{\PYZus{}\PYZus{}init\PYZus{}\PYZus{}}\PY{p}{(}\PY{n+nb+bp}{self}\PY{p}{,} \PY{n}{problem}\PY{p}{)}\PY{p}{:}
                \PY{n+nb+bp}{self}\PY{o}{.}\PY{n}{initial} \PY{o}{=} \PY{n}{problem}\PY{o}{.}\PY{n}{initial}
                \PY{n+nb+bp}{self}\PY{o}{.}\PY{n}{problem} \PY{o}{=} \PY{n}{problem}
                \PY{n+nb+bp}{self}\PY{o}{.}\PY{n}{analizados}  \PY{o}{=} \PY{l+m+mi}{0}
                \PY{n+nb+bp}{self}\PY{o}{.}\PY{n}{goal} \PY{o}{=} \PY{n}{problem}\PY{o}{.}\PY{n}{goal}
        
            \PY{k}{def} \PY{n+nf}{actions}\PY{p}{(}\PY{n+nb+bp}{self}\PY{p}{,} \PY{n}{estado}\PY{p}{)}\PY{p}{:}
                \PY{k}{return} \PY{n+nb+bp}{self}\PY{o}{.}\PY{n}{problem}\PY{o}{.}\PY{n}{actions}\PY{p}{(}\PY{n}{estado}\PY{p}{)}
            
            \PY{k}{def} \PY{n+nf}{\PYZus{}is\PYZus{}valid}\PY{p}{(}\PY{n+nb+bp}{self}\PY{p}{,} \PY{n}{estado}\PY{p}{)}\PY{p}{:}
                \PY{k}{return} \PY{n+nb+bp}{self}\PY{o}{.}\PY{n}{problem}\PY{o}{.}\PY{n}{\PYZus{}is\PYZus{}valid}\PY{p}{(}\PY{n}{estado}\PY{p}{)}
        
            \PY{k}{def} \PY{n+nf}{result}\PY{p}{(}\PY{n+nb+bp}{self}\PY{p}{,} \PY{n}{estado}\PY{p}{,} \PY{n}{accion}\PY{p}{)}\PY{p}{:}
                \PY{k}{return} \PY{n+nb+bp}{self}\PY{o}{.}\PY{n}{problem}\PY{o}{.}\PY{n}{result}\PY{p}{(}\PY{n}{estado}\PY{p}{,} \PY{n}{accion}\PY{p}{)}
        
            \PY{k}{def} \PY{n+nf}{goal\PYZus{}test}\PY{p}{(}\PY{n+nb+bp}{self}\PY{p}{,} \PY{n}{estado}\PY{p}{)}\PY{p}{:}
                \PY{n+nb+bp}{self}\PY{o}{.}\PY{n}{analizados} \PY{o}{+}\PY{o}{=} \PY{l+m+mi}{1}
                \PY{k}{return} \PY{n+nb+bp}{self}\PY{o}{.}\PY{n}{problem}\PY{o}{.}\PY{n}{goal\PYZus{}test}\PY{p}{(}\PY{n}{estado}\PY{p}{)}
        
            \PY{k}{def} \PY{n+nf}{coste\PYZus{}de\PYZus{}aplicar\PYZus{}accion}\PY{p}{(}\PY{n+nb+bp}{self}\PY{p}{,} \PY{n}{estado}\PY{p}{,} \PY{n}{accion}\PY{p}{)}\PY{p}{:}
                \PY{k}{return} \PY{n+nb+bp}{self}\PY{o}{.}\PY{n}{problem}\PY{o}{.}\PY{n}{coste\PYZus{}de\PYZus{}aplicar\PYZus{}accion}\PY{p}{(}\PY{n}{estado}\PY{p}{,}\PY{n}{accion}\PY{p}{)}
\end{Verbatim}


    \begin{Verbatim}[commandchars=\\\{\}]
{\color{incolor}In [{\color{incolor}40}]:} \PY{k}{def} \PY{n+nf}{resuelve\PYZus{}y\PYZus{}muestra\PYZus{}metricas}\PY{p}{(}\PY{n}{problema}\PY{p}{,} \PY{n}{algoritmo}\PY{p}{,} \PY{n}{h}\PY{o}{=}\PY{k+kc}{None}\PY{p}{)}\PY{p}{:}
             
             \PY{k}{if} \PY{n}{h}\PY{p}{:} \PY{n}{sol}\PY{o}{=} \PY{n}{algoritmo}\PY{p}{(}\PY{n}{problema}\PY{p}{,} \PY{n}{h}\PY{p}{)}\PY{o}{.}\PY{n}{solution}\PY{p}{(}\PY{p}{)}
             \PY{k}{else}\PY{p}{:} \PY{n}{sol}\PY{o}{=} \PY{n}{algoritmo}\PY{p}{(}\PY{n}{problema}\PY{p}{)}\PY{o}{.}\PY{n}{solution}\PY{p}{(}\PY{p}{)}
                 
             \PY{n+nb}{print}\PY{p}{(}\PY{l+s+s2}{\PYZdq{}}\PY{l+s+s2}{Longitud de la solución: }\PY{l+s+si}{\PYZob{}0\PYZcb{}}\PY{l+s+s2}{. Nodos analizados: }\PY{l+s+si}{\PYZob{}1\PYZcb{}}\PY{l+s+s2}{\PYZdq{}}\PY{o}{.}\PY{n}{format}\PY{p}{(}\PY{n+nb}{len}\PY{p}{(}\PY{n}{sol}\PY{p}{)}\PY{p}{,}\PY{n}{problema}\PY{o}{.}\PY{n}{analizados}\PY{p}{)}\PY{p}{)}
\end{Verbatim}


    \begin{Verbatim}[commandchars=\\\{\}]
{\color{incolor}In [{\color{incolor}41}]:} \PY{n}{problema\PYZus{}misioneros} \PY{o}{=} \PY{n}{ProblemaMisioneros}\PY{p}{(}\PY{p}{(}\PY{l+m+mi}{3}\PY{p}{,} \PY{l+m+mi}{3}\PY{p}{,} \PY{l+m+mi}{0}\PY{p}{)}\PY{p}{,} \PY{p}{(}\PY{l+m+mi}{0}\PY{p}{,} \PY{l+m+mi}{0}\PY{p}{,} \PY{l+m+mi}{1}\PY{p}{)}\PY{p}{)}
\end{Verbatim}


    \paragraph{-\/-\/-\/-\/-\/-\/-\/-\/-\/-\/-\/-\/-\/-\/-\/-\/-\/-\/-\/-\/-\/-\/-\/-\/-\/-\/-\/-\/-\/-\/-\/-\/-\/-\/-\/-\/-\/-\/-\/-\/-\/-\/-\/-\/-\/-\/-\/-\/-\/-\/-\/-\/-\/-\/-\/-\/-\/-\/-\/-\/-\/-\/-\/-\/-\/-\/-\/-\/-\/-\/-\/-\/-\/-\/-\/-\/-\/-\/-\/-\/-\/-\/-\/-\/-\/-\/-\/-\/-\/-\/-\/-\/-\/-\/-\/-\/-\/-\/-\/-\/-\/-\/-\/-\/-\/-\/-\/-\/-\/-\/-\/-\/-\/-\/-\/-\/-\/-\/-\/-\/-\/-\/-\/-\/-\/-\/-\/-\/-\/-\/-\/-\/-\/-\/-\/-\/-\/-\/-\/-\/-\/-\/-\/-\/-\/-\/-\/-\/-\/-\/-}\label{section}

    \subparagraph{Búsqueda en anchura sin control de
repetidos:}\label{buxfasqueda-en-anchura-sin-control-de-repetidos}

    \begin{Verbatim}[commandchars=\\\{\}]
{\color{incolor}In [{\color{incolor}11}]:} \PY{o}{\PYZpc{}\PYZpc{}}\PY{k}{time}
         breadth\PYZus{}first\PYZus{}tree\PYZus{}search(estado).solution()
\end{Verbatim}


    \begin{Verbatim}[commandchars=\\\{\}]
CPU times: user 113 ms, sys: 772 µs, total: 114 ms
Wall time: 113 ms

    \end{Verbatim}

\begin{Verbatim}[commandchars=\\\{\}]
{\color{outcolor}Out[{\color{outcolor}11}]:} [('2c', (0, 2)),
          ('1c', (0, 1)),
          ('2c', (0, 2)),
          ('1c', (0, 1)),
          ('2m', (2, 0)),
          ('1m1c', (1, 1)),
          ('2m', (2, 0)),
          ('1c', (0, 1)),
          ('2c', (0, 2)),
          ('1c', (0, 1)),
          ('2c', (0, 2))]
\end{Verbatim}
            
    \begin{Verbatim}[commandchars=\\\{\}]
{\color{incolor}In [{\color{incolor}12}]:} \PY{o}{\PYZpc{}\PYZpc{}}\PY{k}{timeit}
         breadth\PYZus{}first\PYZus{}tree\PYZus{}search(estado).solution()
\end{Verbatim}


    \begin{Verbatim}[commandchars=\\\{\}]
10 loops, best of 3: 67.3 ms per loop

    \end{Verbatim}

    \begin{Verbatim}[commandchars=\\\{\}]
{\color{incolor}In [{\color{incolor}42}]:} \PY{n}{problema\PYZus{}metricas}   \PY{o}{=} \PY{n}{ProblemaConMetricas}\PY{p}{(}\PY{n}{problema\PYZus{}misioneros}\PY{p}{)}
         \PY{n}{resuelve\PYZus{}y\PYZus{}muestra\PYZus{}metricas}\PY{p}{(}\PY{n}{problema\PYZus{}metricas}\PY{p}{,} \PY{n}{breadth\PYZus{}first\PYZus{}tree\PYZus{}search}\PY{p}{)}
\end{Verbatim}


    \begin{Verbatim}[commandchars=\\\{\}]
Longitud de la solución: 11. Nodos analizados: 11878

    \end{Verbatim}

    \subparagraph{Búsqueda en anchura con control de
repetidos:}\label{buxfasqueda-en-anchura-con-control-de-repetidos}

    \begin{Verbatim}[commandchars=\\\{\}]
{\color{incolor}In [{\color{incolor}50}]:} \PY{o}{\PYZpc{}\PYZpc{}}\PY{k}{time}
         breadth\PYZus{}first\PYZus{}graph\PYZus{}search(estado).solution()
\end{Verbatim}


    \begin{Verbatim}[commandchars=\\\{\}]
CPU times: user 154 µs, sys: 6 µs, total: 160 µs
Wall time: 164 µs

    \end{Verbatim}

\begin{Verbatim}[commandchars=\\\{\}]
{\color{outcolor}Out[{\color{outcolor}50}]:} [('2c', (0, 2)),
          ('1c', (0, 1)),
          ('2c', (0, 2)),
          ('1c', (0, 1)),
          ('2m', (2, 0)),
          ('1m1c', (1, 1)),
          ('2m', (2, 0)),
          ('1c', (0, 1)),
          ('2c', (0, 2)),
          ('1c', (0, 1)),
          ('2c', (0, 2))]
\end{Verbatim}
            
    \begin{Verbatim}[commandchars=\\\{\}]
{\color{incolor}In [{\color{incolor}18}]:} \PY{o}{\PYZpc{}\PYZpc{}}\PY{k}{timeit}
         breadth\PYZus{}first\PYZus{}graph\PYZus{}search(estado).solution()
\end{Verbatim}


    \begin{Verbatim}[commandchars=\\\{\}]
10000 loops, best of 3: 79.6 µs per loop

    \end{Verbatim}

    \begin{Verbatim}[commandchars=\\\{\}]
{\color{incolor}In [{\color{incolor}43}]:} \PY{n}{problema\PYZus{}metricas}   \PY{o}{=} \PY{n}{ProblemaConMetricas}\PY{p}{(}\PY{n}{problema\PYZus{}misioneros}\PY{p}{)}
         \PY{n}{resuelve\PYZus{}y\PYZus{}muestra\PYZus{}metricas}\PY{p}{(}\PY{n}{problema\PYZus{}metricas}\PY{p}{,} \PY{n}{breadth\PYZus{}first\PYZus{}graph\PYZus{}search}\PY{p}{)}
\end{Verbatim}


    \begin{Verbatim}[commandchars=\\\{\}]
Longitud de la solución: 11. Nodos analizados: 15

    \end{Verbatim}

    \subparagraph{Búsqueda en profundidad con control de
repetidos:}\label{buxfasqueda-en-profundidad-con-control-de-repetidos}

    \begin{Verbatim}[commandchars=\\\{\}]
{\color{incolor}In [{\color{incolor}51}]:} \PY{o}{\PYZpc{}\PYZpc{}}\PY{k}{time}
         depth\PYZus{}first\PYZus{}graph\PYZus{}search(estado).solution()
\end{Verbatim}


    \begin{Verbatim}[commandchars=\\\{\}]
CPU times: user 151 µs, sys: 6 µs, total: 157 µs
Wall time: 160 µs

    \end{Verbatim}

\begin{Verbatim}[commandchars=\\\{\}]
{\color{outcolor}Out[{\color{outcolor}51}]:} [('1m1c', (1, 1)),
          ('1m', (1, 0)),
          ('2c', (0, 2)),
          ('1c', (0, 1)),
          ('2m', (2, 0)),
          ('1m1c', (1, 1)),
          ('2m', (2, 0)),
          ('1c', (0, 1)),
          ('2c', (0, 2)),
          ('1m', (1, 0)),
          ('1m1c', (1, 1))]
\end{Verbatim}
            
    \begin{Verbatim}[commandchars=\\\{\}]
{\color{incolor}In [{\color{incolor}20}]:} \PY{o}{\PYZpc{}\PYZpc{}}\PY{k}{timeit}
         depth\PYZus{}first\PYZus{}graph\PYZus{}search(estado).solution()
\end{Verbatim}


    \begin{Verbatim}[commandchars=\\\{\}]
10000 loops, best of 3: 78.4 µs per loop

    \end{Verbatim}

    \begin{Verbatim}[commandchars=\\\{\}]
{\color{incolor}In [{\color{incolor}48}]:} \PY{n}{problema\PYZus{}metricas}   \PY{o}{=} \PY{n}{ProblemaConMetricas}\PY{p}{(}\PY{n}{problema\PYZus{}misioneros}\PY{p}{)}
         \PY{n}{resuelve\PYZus{}y\PYZus{}muestra\PYZus{}metricas}\PY{p}{(}\PY{n}{problema\PYZus{}metricas}\PY{p}{,} \PY{n}{depth\PYZus{}first\PYZus{}graph\PYZus{}search}\PY{p}{)}
\end{Verbatim}


    \begin{Verbatim}[commandchars=\\\{\}]
Longitud de la solución: 11. Nodos analizados: 12

    \end{Verbatim}

    \subparagraph{Búsqueda en profundidad sin control de
repetidos:}\label{buxfasqueda-en-profundidad-sin-control-de-repetidos}

    \begin{Verbatim}[commandchars=\\\{\}]
{\color{incolor}In [{\color{incolor} }]:} \PY{c+c1}{\PYZsh{} \PYZhy{}\PYZhy{}\PYZhy{}\PYZhy{}\PYZhy{}\PYZhy{}\PYZhy{}\PYZhy{}\PYZhy{}\PYZhy{}\PYZhy{}\PYZhy{}\PYZhy{} NO FINALIZA \PYZhy{}\PYZhy{}\PYZhy{}\PYZhy{}\PYZhy{}\PYZhy{}\PYZhy{}\PYZhy{}\PYZhy{}\PYZhy{}\PYZhy{}\PYZhy{}\PYZhy{}\PYZhy{}}
        \PY{c+c1}{\PYZsh{} \PYZpc{}\PYZpc{}time}
        \PY{c+c1}{\PYZsh{} depth\PYZus{}first\PYZus{}tree\PYZus{}search(estado).solution()}
\end{Verbatim}


    \begin{Verbatim}[commandchars=\\\{\}]
{\color{incolor}In [{\color{incolor} }]:} \PY{c+c1}{\PYZsh{} \PYZhy{}\PYZhy{}\PYZhy{}\PYZhy{}\PYZhy{}\PYZhy{}\PYZhy{}\PYZhy{}\PYZhy{}\PYZhy{}\PYZhy{}\PYZhy{}\PYZhy{} NO FINALIZA \PYZhy{}\PYZhy{}\PYZhy{}\PYZhy{}\PYZhy{}\PYZhy{}\PYZhy{}\PYZhy{}\PYZhy{}\PYZhy{}\PYZhy{}\PYZhy{}\PYZhy{}\PYZhy{}}
        \PY{c+c1}{\PYZsh{} \PYZpc{}\PYZpc{}timeit}
        \PY{c+c1}{\PYZsh{} depth\PYZus{}first\PYZus{}tree\PYZus{}search(estado).solution()}
\end{Verbatim}


    \begin{Verbatim}[commandchars=\\\{\}]
{\color{incolor}In [{\color{incolor} }]:} \PY{c+c1}{\PYZsh{} \PYZhy{}\PYZhy{}\PYZhy{}\PYZhy{}\PYZhy{}\PYZhy{}\PYZhy{}\PYZhy{}\PYZhy{}\PYZhy{}\PYZhy{}\PYZhy{}\PYZhy{} NO FINALIZA \PYZhy{}\PYZhy{}\PYZhy{}\PYZhy{}\PYZhy{}\PYZhy{}\PYZhy{}\PYZhy{}\PYZhy{}\PYZhy{}\PYZhy{}\PYZhy{}\PYZhy{}\PYZhy{}}
        \PY{c+c1}{\PYZsh{} problema\PYZus{}metricas   = ProblemaConMetricas(problema\PYZus{}misioneros)}
        \PY{c+c1}{\PYZsh{} resuelve\PYZus{}y\PYZus{}muestra\PYZus{}metricas(problema\PYZus{}metricas, depth\PYZus{}first\PYZus{}tree\PYZus{}search)}
\end{Verbatim}


    \subsubsection{d) ¿Qué algoritmo consideras mejor? Razonad la
respuesta:}\label{d-quuxe9-algoritmo-consideras-mejor-razonad-la-respuesta}

    Está claro en esta situación que el algoritmo de búsqueda en profundidad
sin control de repetidos no puede ser el mejor por no ser ni siquiera
válido. De los restantes observamos que la búsqueda en anchura sin
control de repetidos genera también una enorme cantidad de nodos
explorados (12000 frente a los 12 de la búsqueda en profundidad con
control de repetidos) y que a su vez su tiempo de ejecución es también
del orden de mil veces mayor a su versión con control de ciclos. Esto es
debido a que gracias a la representación tan sencilla del problema (con
a penas tres valores enteros) las comprobaciones son de bajísimo coste
algorítmico.

Por tanto, para este problema queda patente que el control de ciclos
incrementa el coste de las iteraciones ridículamente y reduce en órdenes
exponenciales (en función de la profundidad de las soluciones
encontradas) el número de nodos analizados. De los dos alogritmos con
control de repeticiones el de búsqueda en profundidad parece ser el
mejor debido a la existencia de múltiples soluciones distribuidas por el
árbol de exploración. La existencia de soluciones a baja profundidad
hace la búsqueda en anchura también una opción válida y eficaz.

    \subsubsection{Ejercicio opcional. Define alguna heurística y estudia
las propiedades del algoritmo
A*}\label{ejercicio-opcional.-define-alguna-heuruxedstica-y-estudia-las-propiedades-del-algoritmo-a}

    La lógica parece indicarnos que cuantos más misioneros y caníbales haya
en la orilla derecha más cerca de la solución estamos. Como bien se
explica en el temario de la asignatura, dentro de las propias
condiciones del problema se especifica que no se puede llevar a más de
dos personas en la lancha. Es por esto que para llevar k personas de la
orilla izquierda a la derecha necesitaremos como mínimo k/2 viajes en la
lancha (de hecho generalmente podríamos crear una heurística más
informada gracias a que únicamente tenemos una balsa y, por tanto, los
viajes de vuelta serán imprescindibles en ciertos casos). Necesitar como
mínimo k/2 viajes (con k el número de personas en la orilla izquierda)
nos hace ver que la heurística definida es en efecto consistente.

    \begin{Verbatim}[commandchars=\\\{\}]
{\color{incolor}In [{\color{incolor}29}]:} \PY{c+c1}{\PYZsh{}\PYZsh{} Heurística elegida: (Número de personas en la orilla izquierda) / (2 = Capacidad de la barca)}
         \PY{k}{def} \PY{n+nf}{heuristica}\PY{p}{(}\PY{n}{node}\PY{p}{)}\PY{p}{:}
             \PY{n}{state} \PY{o}{=} \PY{n}{node}\PY{o}{.}\PY{n}{state}
             \PY{k}{return} \PY{p}{(}\PY{n}{state}\PY{p}{[}\PY{l+m+mi}{0}\PY{p}{]} \PY{o}{+} \PY{n}{state}\PY{p}{[}\PY{l+m+mi}{1}\PY{p}{]}\PY{p}{)}\PY{o}{/}\PY{l+m+mi}{2}
\end{Verbatim}


    \begin{Verbatim}[commandchars=\\\{\}]
{\color{incolor}In [{\color{incolor}30}]:} \PY{n}{astar\PYZus{}search}\PY{p}{(}\PY{n}{problema\PYZus{}misioneros}\PY{p}{,} \PY{n}{heuristica}\PY{p}{)}\PY{o}{.}\PY{n}{solution}\PY{p}{(}\PY{p}{)}
\end{Verbatim}


\begin{Verbatim}[commandchars=\\\{\}]
{\color{outcolor}Out[{\color{outcolor}30}]:} [('1m1c', (1, 1)),
          ('1m', (1, 0)),
          ('2c', (0, 2)),
          ('1c', (0, 1)),
          ('2m', (2, 0)),
          ('1m1c', (1, 1)),
          ('2m', (2, 0)),
          ('1c', (0, 1)),
          ('2c', (0, 2)),
          ('1c', (0, 1)),
          ('2c', (0, 2))]
\end{Verbatim}
            
    \paragraph{Estudiamos ahora las propiedades de la heurística
dada}\label{estudiamos-ahora-las-propiedades-de-la-heuruxedstica-dada}

    \begin{Verbatim}[commandchars=\\\{\}]
{\color{incolor}In [{\color{incolor}46}]:} \PY{o}{\PYZpc{}\PYZpc{}}\PY{k}{time}
         astar\PYZus{}search(estado, heuristica).solution()
\end{Verbatim}


    \begin{Verbatim}[commandchars=\\\{\}]
CPU times: user 470 µs, sys: 24 µs, total: 494 µs
Wall time: 501 µs

    \end{Verbatim}

\begin{Verbatim}[commandchars=\\\{\}]
{\color{outcolor}Out[{\color{outcolor}46}]:} [('1m1c', (1, 1)),
          ('1m', (1, 0)),
          ('2c', (0, 2)),
          ('1c', (0, 1)),
          ('2m', (2, 0)),
          ('1m1c', (1, 1)),
          ('2m', (2, 0)),
          ('1c', (0, 1)),
          ('2c', (0, 2)),
          ('1c', (0, 1)),
          ('2c', (0, 2))]
\end{Verbatim}
            
    \begin{Verbatim}[commandchars=\\\{\}]
{\color{incolor}In [{\color{incolor}49}]:} \PY{o}{\PYZpc{}\PYZpc{}}\PY{k}{timeit}
         astar\PYZus{}search(estado, heuristica).solution()
\end{Verbatim}


    \begin{Verbatim}[commandchars=\\\{\}]
10000 loops, best of 3: 143 µs per loop

    \end{Verbatim}

    \begin{Verbatim}[commandchars=\\\{\}]
{\color{incolor}In [{\color{incolor}47}]:} \PY{n}{problema\PYZus{}metricas}   \PY{o}{=} \PY{n}{ProblemaConMetricas}\PY{p}{(}\PY{n}{problema\PYZus{}misioneros}\PY{p}{)}
         \PY{n}{resuelve\PYZus{}y\PYZus{}muestra\PYZus{}metricas}\PY{p}{(}\PY{n}{problema\PYZus{}metricas}\PY{p}{,} \PY{n}{astar\PYZus{}search}\PY{p}{,} \PY{n}{heuristica}\PY{p}{)}
\end{Verbatim}


    \begin{Verbatim}[commandchars=\\\{\}]
Longitud de la solución: 11. Nodos analizados: 14

    \end{Verbatim}

    Los resultados muestran que la búsqueda ciega en profundidad analiza
menos nodos que la búsqueda guiada por nuestra heurística. Esto, junto
con el mayor coste algorítmico causado por el añadido de complejidad de
la heurística hace que también en los tiempos de ejecución se refleje un
empeoramiento con respecto a los métodos de búsqueda no informados.

La conclusión que podemos obtener es que, dada la escasa complejidad (en
cuanto a tamaño del árbol de exploración) del problema y dada la
existencia de soluciones a baja profundidad (con un bajo número de
operaciones), el uso de una heurística si bien no supone un
empeoramiento sustancial del tiempo ni memoria empleada en la resolución
del problema (gracias a que la heurística es de cómputo en coste
constante y bajo), si que es innecesaria para este problema en concreto.

Si en vez de tratar el problema de los misioneros (3 misioneros y 3
caníbales) tratásemos el problema de las Jornadas Mundiales de la
Juventud (JMJ, con 3000 misioneros y 3000 jóvenes caníbales) el uso de
una heurística sería absolutamente imprescindible y los métodos de
búsqueda ciega resultarían inaplicables.

    \subparagraph{Autores: Francisco Javier Blázquez Martínez, Boris
Carballa Corredoira, Juan Carlos Villanueva
Quirós}\label{autores-francisco-javier-bluxe1zquez-martuxednez-boris-carballa-corredoira-juan-carlos-villanueva-quiruxf3s}


    % Add a bibliography block to the postdoc
    
    
    
    \end{document}
