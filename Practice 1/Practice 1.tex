\documentclass[final]{CSP}
\usepackage[utf8]{inputenc}
\usepackage[spanish]{babel}
\usepackage{amssymb}
\usepackage{changepage}
\usepackage{hyperref}
\usepackage{dramatist}
\usepackage[margin=0.8in]{geometry}


\begin{document}

\begin{frontmatter}

\title{Práctica 1 Inteligencia Artificial}

\author{Francisco Javier Blázquez Martínez}
%\author[mymainaddress,mysecondaryaddress]{Francisco Javier Blázquez Martínez}
%\author[mysecondaryaddress]{Global Customer Service\corref{mycorrespondingauthor}}
%\cortext[mycorrespondingauthor]{Corresponding author}
%\ead{support@Icm.com}
%\address[mymainaddress]{1600 John F Kennedy Boulevard, Philadelphia}
%\address[mysecondaryaddress]{360 Park Avenue South, New York}

\begin{keyword}\rm
\begin{adjustwidth}{2cm}{2cm}{\itshape\textbf{Objetivo:}}  
Esta práctica se realiza en el marco de la asignatura \textit{Inteligencia Artificial} del
cuarto curso del doble grado en matemáticas e ingeniería informática de la universidad 
complutense de Madrid. Se presenta una primera introducción a la inteligencia artificial,
sus conceptos y sus campos de estudio y aplicación más comunes en la actualidad. 
\end{adjustwidth}
\end{keyword}

%\begin{abstract}\rm
%\begin{adjustwidth}{2cm}{2cm}{\itshape\textbf{Abstract:}} 
%The abstract should summarize the contents of the paper and should contain at least 100 and at most 300 words. It should be set in 12-point font size. There should be a space before of 18-point and after of 60-point
%\end{adjustwidth}
%\end{abstract}
\end{frontmatter}

\vspace{7mm}
\setlength\parindent{0cm}

\section{``El largo camino de la IA específica a la IA general''\footnote{Disponible en: \url{https://yungranpasoparalahumanidad.blogspot.com/2019/02/la-inteligencia-artificial-ramon-lopez.html}} por Ramon López de Mántaras Badia}

¿En qué situación se encuentra la IA actualmente? Este es el primer punto que trata de aclarar
el artículo. El autor es sumamente claro, actualmente se desarrollan con bastante éxito ciertas
Inteligencias artificiales de las que hemos denominado en clase de tipo débil. Focalizadas en
una única tarea sumamente específica. Sin embargo, la cuestión fundamental, que permitiría 
desarrolar IIAA de tipo general, sigue abierta 50 años después. ¿Cómo dotar de sentido común a
las máquinas?.\\

No ser capaz de dotar de sentido común a las máquinas hace que la IA esté tremendamente limitada.
Y, lo que es más, el autor se aventura a afirmar que la IA es de naturaleza distinta y posee
diferencias insalvables con respecto a la inteligencia humana. Entonces, reflexiona el autor,
si aceptamos que son inteligencias de naturaleza distinta, ¿cómo confiamos en delegar decisiones
a esta IA que no conocemos realmente?.\\

Entonces, aceptando lo dicho anteriormente, el problema de la IA no es el posible desarrollo de 
una súper inteligencia y sus repercusiones, sino las posibles repercusiones de delegar ciertas 
tareas a IA de tipo específico. ¿Qué podemos delegar a una IA y en qué medida?. \\

En mi caso, me muestro totalmente de acuerdo con el autor, las IIAA están probando ser herramientas
sumamente útiles en determinadas aplicaciones, más aún cuando se combinan con la inteligencia
humana. Sin embargo, como con toda herramienta, su uso puede servir a distintos fines. Tenemos
que ser conscientes del hecho de que la IA no es un martillo o unos alicates, ni siquiera es una
pistola o un rifle, es una herramienta que dada su complejidad muchas veces es usada sin
ser comprendida realmente y dada su naturaleza puede tener un alcance enorme. Sin embargo, debemos
exigir su regulación a aquellas personas y compañías que quieran hacer uso de esta y siempre medir
con mucha cautela posibles repercusiones a priori ocultas de su aplicación.\\

El autor llega a afirmar que, aunque las máquinas llegaran a probar altas capacidades de diversa
índole e, incluso mostrando capacidades intrínsecamente humanas, habría decisiones que nunca 
delegaría en una máquina. Esta opinión puede ser tachada de excesivamente conservadora o 
desconfiada hacia una IA que cada vez muestra más progresos, pero es una opinión que comparto.



\section{Chatbots y asistentes virtuales}

Para la realización de este apartado he optado por interactuar con el chatbot \textit{Mitsuku}
\footnote{Disponible en: \url{https://pandorabots.com/mitsuku/}}, ganador en varias ediciones 
del prestigioso \textit{Loebner Prize} y con \textit{Martín} \footnote{Disponible en:
\url{https://consulta.renfe.com/martin/main}}, el chatbot especializado en atención al cliente de la
compañia ferroviaria Renfe. Se exponen a continuación las conclusiones obtenidas:

\subsection{Mitsuku}

Dentro de que Mitsuku es un chatbot y, como tal, no es parte de lo que consideraríamos una IA
general, puesto que su único fin y capacidad es conversar con un humano (dejado de lado otras 
muchas cualidades que también consideramos intrínsecas a los seres humanos), no deja de ser 
sorprendente las escasas limitaciones que posee. Ante preguntas concretas pero de muy diversa 
índole Mitsuku con gran frecuencia da respuestas que podríamos considerar propias de un ser humano.
Como por ejemplo las siguientes:

\begin{drama}
  \Character{Mitsuku}{m}
  \Character{Humano}{h}
    
  \hspeaks Hello Mitsuku, do you speak Spanish?
  \mspeaks Hello there. Un poquito. \\
  ...
  \hspeaks Mitsuku, have you ever been in a boat?
  \mspeaks Erm, Yes. I often visit large cruise liners and relax on deck. \\
\end{drama}

Técnicas como la inclusión de expresiones del lenguaje coloquial acercan más incluso a este 
chatbot a lo que sería una conversación entre personas adultas (como el uso de ``\textit{Erm}''
en su respuesta, como si requiriera de un tiempo de reflexión. Sin embargo sus limitaciones quedan
patentes al interactuar y dialogar sobre un tema en concreto. Las conversaciones anteriores 
continuaron de este modo:

\begin{drama}
  \Character{Mitsuku}{m}
  \Character{Humano}{h}
    
  \hspeaks Hello Mitsuku, do you speak Spanish?
  \mspeaks Hello there. Un poquito.
  \hspeaks ¿Pero me entiendes?
  \mspeaks Is it better to be happy in love, or extremely wealthy? \\
  ... 
  \hspeaks Mitsuku, have you ever been in a boat?
  \mspeaks Erm, Yes. I often visit large cruise liners and relax on deck.
  \hspeaks Drinking a mojito maybe?
  \mspeaks Are you thirsty? 
\end{drama}

Y en efecto, al continuar estas conversaciones, Mitsuku parece haber ólvidado por completo el 
contexto en el que realizamos la pregunta, sin tener en cuenta incluso su respuesta inmediatamente
anterior. Otro ejemplo es el siguiente, Mitsuku nos asegura haber visitado Barcelona pero no nos
proporciona respuestas consistentes al profundizar en el tema.

\begin{drama}
  \Character{Mitsuku}{m}
  \Character{Humano}{h}

    \hspeaks Have you ever been in Barcelona?
    \mspeaks Yes I thought it was a very beautiful city with some lovely buildings.
    \hspeaks What did you visit in Barcelona?
    \mspeaks I was constructed at St Jimmy's in Leeds.
\end{drama}

\subsection{Martín}

Martín, a diferencia de Mitsuku, tiene un propósito y un ámbito de conversación claro y específico.
Es sorprendentemente útil cuando se le solicita información relativa a próximos billetes a 
distintos destinos, precios, posibilidades de cambio y anulación de billetes, cómo llegar a
las principales estaciones de tren... Sin embargo, fuera de su ámbito las respuestas se demoran
mucho más y, en gran cantidad de casos, nunca llegan. Incluimos ahora un diálogo qu creemos 
interesante:

\begin{drama}
  \Character{Martín}{m}
  \Character{Humano}{h}

    \hspeaks ¿Cómo puedo viajar de Madrid a Barcelona?
    \mspeaks Aquí tienes un enlace [...]. 
\end{drama}

Martín interpreta correctamente nuestra respuesta y nos ofrece múltiples enlaces para obtener
nuestro billete a Barcelona en el mismo día, en trenes de media y larga distancia o AVE. Además,
nos proporciona gran cantidad de enlaces adicionales que abarcan desde transporte público en
las ciudades de origen y destino, hasta meteorología en estas y ubicación de las estaciones de
tren. Sin embargo, sus limitaciones quedan patentes al profundizar en la conversación.

\begin{drama}
  \Character{Martín}{m}
  \Character{Humano}{h}

    \hspeaks ¿Has estado tú en Barcelona? ¿Crees que merece la pena la visita?
    \mspeaks (sin respuesta)
    \hspeaks ¿Has estado tú en Barcelona?
    \mspeaks No he salido nunca de mi disco. Barcelona es [toneladas de datos concretos].
    \hspeaks Jo que pena, ¿no te sientes sólo en tu disco?
    \mspeaks Me temo que no tengo respuesta para eso.
\end{drama}

Como vemos, no trata de ocultar que no es un ser humano real y evita responder a preguntas que
sobrepasan su ámbito. En su ámbito sin embargo, proporciona respuestas tan detalladas y con tal
cantidad de información que es claro que estas no provienen de un ser humano.

\subsection{Posible pregunta para el Loebner Prize}

\textbf{\textit{Hola Martín, mi hijo adolescente no quiere pasar tiempo con la 
familia. ¿Qué crees que le pasa?}}\\
Me he decantado por una pregunta de este tipo porque tiene un alto componente social y de
conocimiento (en gran parte no escrito, sino implícito) de las relaciones sociales del día
a día. Creo que todos sabríamos responder correctamente a una pregunta de este tipo, podríamos
expresar la falta de necesidad de preocupación, explicar que va asociado con la edad ese 
comportamiento o que el jovenzuelo únicamente quiere salir, conocer chicas o jugar al fútbol
y videojuegos. Sin embargo estas son las respuestas obtenidas por los chatbots:
\begin{drama}
  \Character{Martín}{m}
  \Character{Humano}{h}
  \Character{Mitsuku}{mi}

    \hspeaks Hola Martín, mi hijo adolescente no quiere pasar tiempo con la familia. ¿Qué crees 
             que le pasa?
    \mspeaks Tanto gusto, ¡hola Martín! \textit{(confusión, respuesta sin sentido)}
    \mispeaks How many children do you have? Would you like some more one day? 
              \textit{(incoherencia)}
\end{drama}

\subsection{Conclusiones}

Mis conclusiones tras interactuar con estos chatbots han sido, a grandes rasgos, que poseen una
gran versatilidad para entender la formulación de preguntas concretas en lo referido a estructuras
sintácticas e incluso he llegado a ver un buen conocimiento del lenguaje pero, sin embargo, han 
resultado tremendamente incapaces de mantener una conversación sencilla, dando respuestas 
coherentes no únicamente con la pregunta sino con el contexto. Esto me hace pensar que, incluso
superar el test de Turing no es a mi parecer garantía de poseer \textit{Inteligencia} por parte
de una máquina. Tal vez Allan Turing no pensaba en un test al estilo del Loebner Prize, con unas
preguntas de muy distinto tipo prefijadas, sino que pensaba en un test más abstracto en el que
el juez pudiera decidir su próxima pregunta en función también de las respuestas obtenidas. Este
es a mi parecer el auténtico test de Turing a pesar de que entiendo que el formato actual del 
Loebner Prize elimina cierta subjetividad y lo hace más fácilmente evaluable.

\section{Traductor de Google}

Esta famosa herramienta, ampliamente utilizada y con soporte para más de un centenar de idiomas 
es un ejemplo característico de aplicación de los avances en el campo de la Inteligencia 
Artificial (y en la estadística computacional) a una labor particularmente complicada para las
máquinas, como es el estudio del lenguaje y su traducción.\\

Sobre su funcionamiento interno cabe decir primero que la aproximación algorítmica natural, 
mediante la cual se dispone de un diccionario de palabras para asignar a cada cual en un idioma
inicial otra en el idioma deseado y que trataría de traducir cualquier frase o párrafo palabra
por palabra no es válida. Esto es fácil de ver si se observa que existen palabras con varios 
sinónimos, lo que implicaría perder riqueza de léxico, las construcciones sintácticas pueden ser
radicalmente distintas entre idiomas, verbos con distintas conjugaciones y tiempos y, lo que puede
ser lo más difícil de todo, pérdida de contexto.\\

Es por esto que la primera aproximación desarrollada por Google (año 20106) se basaba en un gran
volumen de datos de traducciones y un análisis estadístico de estos datos para que así, el propio
uso de las palabras y determinadas oraciones permitiera inferir el contexto en el que se producía
la traducción (método denominado \textit{Traducción automática estadística}). Otro hecho clave de
esta primera implementación es que los textos siempre eran traducidos al inglés desde su idioma
inicial y entonces, a partir de esta versión en inglés se traducían al idioma final (debido
principalemente a falta de datos suficientes entre otros idiomas para conseguir una traducción 
automática estadística fiable.\\

En el año 2016 Google opta por una implementación alternativa y mejorada de su traductor. Pasa
de la traducción automática estadística a un modelo basado en una red neuronal con técnicas de
deep learning (sólo en determinados idiomas, manteniendo la implementación anterior en los 
restantes). Esto significa una mejora porque el traductor aprende de millones de ejemplos que se
le proporcionan y trata de traducir oraciones enteras, en lugar de traducir independientemente
pequeños bloques de palabras. Esto de nuevo, aumenta el contexto que el traductor es capaz de 
contemplar. Además, se elimina en estos idiomas soportados por la nueva implementación la necesidad
del Inglés como idioma intermedio, lo que reduce la acumulación de errores de traducción. Sin 
embargo, los errores siguen siendo parte de la herramienta. Por ejemplo:

\begin{drama}
  \Character{ESP}{e}
  \Character{ING}{i}

    \espeaks Mi padre está chapado a la antigua.
    \ispeaks My father is old fashioned.
\end{drama}

Proporciona una traducción bastante buena para la complejidad de la oración original. Es claro
que esta traducción requiere entender todo el bloque ``\textit{estar chapado a la antigua}" que,
aunque lo constituyen un total de cinco palabras, tienen un sentido único. Sin embargo, es claro
que la traducción no ha conservado el significado, en inglés da a entender que mi padre está
anticuado o pasado de moda pero no hace referencia a su comportamiento y sus modales antiguos.
De hecho, si realizamos la traducción inversa:

\begin{drama}
  \Character{ESP}{e}
  \Character{ING}{i}

    \ispeaks My father is old fashioned.
    \espeaks Mi padre es anticuado.
\end{drama}

No llegamos al punto inicial, de hecho obtenemos una oración mal formada. En español a diferencia
del idioma inglés, como es por todos conocidos, ser y estar son verbos distintos y el verbo ser
no está bien empleado en la traducción, debería ser estar. De hecho si continuamos traduciendo:

\begin{drama}
  \Character{ESP}{e}
  \Character{ING}{i}

    \espeaks Mi padre es anticuado.
    \ispeaks My father is outdated.
    \espeaks Mi padre esta anticuado.
\end{drama}

Apreciamos aquí también la pérdida de riqueza léxica, nosotros hemos partido de ``\textit{old
fashioned''} y al tratar de hacer la traducción inversa hemos recibido ``\textit{outdated}''. Esto
es causado porque ``\textit{outdated}'' es la de mayor uso (en los datos almacenados por Google).
El problema radica en que siempre que requiramos al traductor una traducción para 
``\textit{anticuado}'' vamos a recibir ``\textit{outdated}'', cuando de requerir esta traducción a
varios traductores o a un mismo trauctor en varias situaciones podríamos obtener varias 
perfectamente válidas. Otros ejemplo de errore del traductor puede ser el siguiente:

\begin{drama}
  \Character{ESP}{e}
  \Character{ING}{i}

    \espeaks Jugar a la petanca es de viejos.
    \ispeaks Playing petanque is old
\end{drama}

Auquí de nuevo el problema está en la construcción de la expresión. ``\textit{Ser de viejos}'' 
es una expresión que no es capaz de traducir. Sin embargo, debo decir que incluso en párrafos
mucho más complejos extraídos de la edición del dia de hoy de un periódico, la traducción entre
Español e Inglés del traductor da muy buenos resultados. Esto se debe principalmente a la gran 
cantidad de datos de traducciones de los que se dispone para estos dos idiomas. Sin embargo, el
traductor de google no alcanza este nivel al traducir a gran cantidad de idiomas Africanos por 
escasez de datos o al Chino o al Koreano por lo radicalmente distinto de las estructuras 
sintácticas.

\section{Robótica}

\subsection{Algunos tipos de robots}

\begin{itemize}
\item \textbf{Manipuladores:}\\
También llamados robots industriales. Un robot manipulador es un robot (en el sentido de  
dispositivo móvil programable) capaz de mover materias, piezas, herramientas, o dispositivos 
especiales, según trayectorias variables, programado para realizar tareas diversas. Son
ampliamente utilizados en industria por su precisión pero siempre en tareas concretas para
las que puedan ser programables. Hay cientos de ejemplos y empresas dedicadas a su venta e 
instalación.

\item \textbf{Humanoides:}\\
Son aquellos robots diseñados para simular la foma y movimiento de un ser humano. Dentro de estos
cabe destacar los denomidados androides, diseñados también para tener un parecido estético con los
humanos. Algunos alcanzan niveles de realidad bastante altos como el robot \textit{Sofía} o el 
robot (o al menos la cabeza robot) \textit{Martin Kelley}.

\item \textbf{Zoomórficos:}\\
Son aquellos que simulan en su forma exterior y movimientos a seres vivos y animales. Su interes
es en ciertos campos mayor al de los robots humanoides pues, a diferencia de estos que para 
emular a los seres humanos son bípedos, estos pueden tener medios de locomoción más estables lo
que los hace idóneos para campos como exploración espacial y exploración de otros terrenos donde
las condiciones son adversas para el acceso de los humanos.
\end{itemize}

\subsection{Análisis del androide Sofía
\footnote{Ver:\url{https://www.youtube.com/watch?v=gB6mVTSr8cQ}}}

Sin duda es un robot humanoide con derecho a ser llamado androide, frente a muchos otros que
únicamente parecen una abstracción de los seres humanos y su movilidad, este cuida hasta cierto
grado también la estética. Sus respuestas a las preguntas de Pablo Motos son realmente avanzadas
para ser un robot pero sin duda están preparadas de antemano para causar una mayor impresión, 
son demasiado elaboradas.\\

Creo que por esto no refleja bien la realidad, porque viendo el vídeo da la impresión de que el
único escollo para conseguir pasar el test de Turing fuerte, que involucra apariencia externa
e interactuación, es la apariencia externa, cuando son las dos. Y más aún la capacidad de 
interactuar que la estética, porque requiere pensar. Es más, si alguna vez veo a un robot cantar
a Rosalía habiéndolo aprendido por su propia curiosidad, yo le concederé a ese robot mi particular
visto bueno al test de Turing, pero este vídeo es un montaje. Sí que podemos extraer de él sin 
embargo que el punto en el que se encuentra la robótica con la IA es más avanzado de lo que mucha
gente cree y que al mismo tiempo está muchísimo más lejos de la IA general de lo que mucha gente
cree.

%\bibliographystyle{bibft}\it
%\bibliography{bibfile}
%\appendix
%\section{}
%\label{}

\end{document}



% https://en.wikipedia.org/wiki/Google_Translate
% https://es.wikipedia.org/wiki/Traducción_automática_estadística
% https://en.wikipedia.org/wiki/Google_Neural_Machine_Translation
%https://es.wikipedia.org/wiki/Robot_humanoide
%https://sites.google.com/a/unitecnica.net/dmhenao/articulos/zoomorficos